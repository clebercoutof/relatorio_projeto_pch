% ----------------------------------------------------------
% Introdução 
% Capítulo sem numeração, mas presente no Sumário
% ----------------------------------------------------------

\chapter[Introdução]{Introdução}
\addcontentsline{toc}{chapter}{Introdução}

O avanço da tecnologia está relacionado diretamente com a demanda de energia elétrica. A inserção de cada vez mais aparelhos eletrônicos, automatização de antigos processos e criação de novas tecnologias como carros elétricos implica um aumento constante na demanda de energia elétrica. “Em 2030, estima-se um consumo de energia elétrica entre 950 e 1.250 TWh/ano, sendo que o consumo atual situa-se em torno de 405 TWh” (ANEEL, Atlas de Energia Elétrica no Brasil 2006).

A perspectiva do aumento da demanda faz com que seja necessário um investimento maior no setor energético, de acordo com Bronzati essa grande diferença entre a demanda de 2030 que a demanda atual “exigirá investimentos pesados na expansão da oferta de energia elétrica. No caso deste fornecimento ser realizado por usinas hidrelétricas, mesmo com uma instalação adicional de 120 mil MW, o que eleva para 80\% o uso do potencial, ainda assim poderia não ser suficiente para atender a demanda em 2030.

Mesmo com o aumento do uso do potencial hídrico, é notável que há uma necessidade de diversificação da matriz energética, onde essa diversificação deve buscar a inserção de fontes renováveis. Pequenas Centrais Hidrelétricas representam se mostram uma alternativa muito viável, possibilitando uma geração próxima da carga, um impacto ambiental menor que as grandes usinas, além de um maior custo.



\section{Objetivo}\label{sec:obj}
Projetar uma PCH e realizar seu estudo de viabilidade.

\subsection{Objetivos Específicos}
\begin{itemize}
\item Escolher uma bacia hidrográfica que não possua nenhuma usina instalada;
\item Determinar a potência gerada ao ano;
\item Determinar as máquinas utilizadas;
\item Descrever o maquinário e estruturas auxiliares;
\item Realizar o estudo de payback;
\item Avaliar o projeto técnico e financeiramente;

\end{itemize}
\section{Fundamentação teórica}
Para realização do trabalho foram necessários os seguintes conhecimentos relativos à redes e comunicações.
\subsection{Comunicação Serial}
Comunicação serial é a nomenclatura atribuída ao processo da troca de dados de forma sequencial por meio de um canal ou barramento específico para comunicação . Tais dados são bytes de informação, transmitidos bit a bit, por meio de uma porta serial.	

Os protocolos de comunicação serial representam o modo que a mensagem é transmitida, definindo a forma como os bytes serão ordenados de modo a garantir que a mensagem seja transmitida.
O padrão de comunicação diz respeito à estrutura física da comunicação, fazendo referência aos padrões elétricos envolvidos e quantidade de vias utilizadas.
\subsection{Protocolo MODBUS}
O protocolo MODBUS O protocolo Modbus é uma estrutura de mensagem aberta desenvolvida pela Modicon na década de 70. De acordo com a MODBUS Organization o protocolo MODBUS É:  “É um protocolo de camada de aplicação posicionado no nível 7 do modelo OSI , que fornece comunicação cliente / servidor entre dispositivos conectados em diferentes tipos de barramentos ou redes”. 

Consiste em um protocolo de solicitação/resposta , fornecendo serviços especificados por códigos de função. Os códigos de função de MODBUS são elementos das PDUs(Protocol Data Unit ou unidade de dados de protocolo) de solicitação / resposta de MODBUS.
É largamente utilizado em automação industrial, sendo transmitido por protocolos físicos como o RS232/RS485 e Ethernet TCP/IP. 

\subsubsection{Modo de transmissão RTU}
RTU é a sigla para Remote Terminal Unit, nesse modo de transmissão cada mensagem de dados de 8 bits contém dois caracteres hexadecimais de 4 bits, isso faz com que o processamento de dados seja mais rápido se comparado com a transmissão por 2 caracteres Ascii de 4 bits, já que a densidade de caracteres é maior. Apresenta 8 bits disponíveis para o endereço, 8 para função, 0 a 252 bits de data e 16 bits para o CRC Check, como mostrado na figura \ref{fig:rtu_packet} a seguir:
\begin{figure}[h]
	\centering
	\includegraphics[width=.5\textwidth]{rtu_packet.png}
	\caption{Pacote da comunicação RTU}
	\label{fig:rtu_packet}
	%\source{Fornecido junto com os pontos}
\end{figure} 
\subsubsection{Modo de transmissão TCP}
O modo de transmissão TCP é uma aplicação do protocolo MODBUS baseado em TCP/IP, consequentemente utilizando a conexão ethernet. Para a pilha de comunicação, nesse protocolo é adicionado ao quadro um cabeçalho MBAP (MODBUS Application Protocol), deitando o modelo de mensagem como a figura \ref{fig:tcp_packet}:

\begin{figure}[h]
	\centering
	\includegraphics[width=.6\textwidth]{tcp_packet.png}
	\caption{Pacote da comunicação TCP}
	\label{fig:tcp_packet}
	%\source{Fornecido junto com os pontos}
\end{figure}
Esse novo cabeçalho tem 7 bytes de tamanho sendo composto por os seguintes elementos:
\begin{itemize}
	\item Transaction identifier: usado para identificação da resposta para a transação (2 bytes);
	\item Protocol identifier: 0 (zero) indica Modbus (2 bytes);
	\item Length: contagem de todos os próximos bytes (2 bytes);
	\item Unit identifier: utilizado para identificar o escravo remoto em uma rede Modbus RTU (1 byte).
	
\end{itemize}

O Modbus  TCP nao utiliza um byte de checkagem ao final da mensagem, pois o pŕoprio frame ethernet já apresenta uma checagem do tipo CRC-32. Na comunicação o cliente Modbus TCP inicia a conexão com o servidor para enviar as requisições, onde a porta padrão para a conexão com os servidores é a TCP 502.
